\documentclass[11pt]{article}

\input{macros}
\date{Formal Verification of Critical Appicaitons -- 2021/2022}

\begin{document}
 
% --------------------------------------------------------------
%                         Start here
% --------------------------------------------------------------
 
\title{Part 1: Modelling with time}
\author{Jos\'{e} Proen\c{c}a \& David Pereira \& Eduardo Tovar
\\
\{pro,drp,emt\}@isep.ipp.pt}
%\\
%Arquitectura e C\'alculo -- 2015/2016} 
 
\maketitle

\descrbox{To do}{
  Develop the requested Uppaal specification and requirements, and produce a report in PDF, including screenshots of the requested timed automata.  
}

\descrbox{What to submit}{
  The \emph{PDF report} \textbf{and} the developed \emph{Uppaal models}.
}

\descrbox{How to submit using git}{
  \begin{enumerate}[itemsep=1pt,leftmargin=20pt]
    \item Create a private git repository in your favouring host (e.g., github or bitbucket).
    \item \textbf{Name it \bash{FVOCA22-g<group number>}.}
    \item Add \bash{pro@isep.ipp.pt} and \bash{drp@isep.ipp.pt} as members of the group (read-permissions are enough).
    \item Include all the files to be submitted in the repository.
  \end{enumerate}
  Note that \textbf{all students should push commits.}
}

\descrbox{Deadline}{
  tba @ 23:59 (Sunday) 
}




% \myparagraph{Deadline part II:} 31 May 2017 @ 23h59 (Wednesday)
 
% \section*{Part I - Real time}

%The train gate example is distributed with Uppaal. It is a railway control system which
% - controls access to a bridge for several trains.
%
%BRIDGE: a critical shared resource that may be accessed only by one train at a time.
%SYSTEM: a number of trains (assume 4 for this example) + a controller.
%
%TRAIN: sends approach, waits 10 secs for stop! signal,
% - not stopped: after 10 more it reaches the gate/bridge;
% - stopped: waits for a go! signal - takes 7-15 sec to reach the cross after go!.
% - sends a leave! signal after passing.
% - after reaching the cross - 3-5 sec to cross.
% (5 locations: Safe, Appr-oaching, Stop-ping, Start-ting, and Cross-sing)
%
%GATE: syncs with queue/contr and trains
% - Can be free or occupied,
% - starts Free, becomes Occupied
%CONTROLER/QUEUE: (not for 2 trains)
%
%Start has the invariant x <= 15 and its outgoing transition has the constraint x >= 7
%
%A train can not be stopped instantly and restarting also takes time. Therefor, there are timing constraints on the trains before entering the bridge. When approaching, a train sends a appr! signal. Thereafter, it has 10 time units to receive a stop signal. This allows it to stop safely before the bridge. After these 10 time units, it takes further 10 time units to reach the bridge if the train is not stopped. If a train is stopped, it resumes its course when the controller sends a go! signal to it after a previous train has left the bridge and sent a leave! signal.

%-----------------------------------------------------------------------------
%\begin{exercise} \label{ex:train}
%\textbf{[Train bridge]}
%Consider a train bridge over a river shared by multiple trains. In this exercise we will consider only 2 trains. Only one train can cross the bridge at a time - a \emph{gate} controls who is allowed to cross the bridge at a given time. The desired model of the train bridge has the following extra requirements:
%%
%\begin{itemize}
%  \item a Train notifies the Gate when it approaches the bridge;
%  \item the Gate has 10 time units to notify the Train to stop (if another train is crossing the bridge);
%  \item if the Gate does not send a stop notification within 10 time units, the Train must reach the bridge in another 10 time units;
%  \item each Train takes between 3 and 5 time units to cross the bridge -- after that it sends a notification to the Gate that it is leaving;
%  \item if a Train is told to stop, we assume it will take between 7 and 15 time units to reach the bridge;
%\end{itemize}
%\end{exercise}

\section*{Motor controller in a Railway system}

A railway company produces signalling systems that are used in critical systems. These must provide enough evidence over their reliability and correctness over time to comply with the heavy certification processes.

This assignment is a simplification of on ongoing use-case of an European project, depicted in \cref{fig:global-architecture}.
In this use-case a critical motor that rotates left and right interacts with a controller. This controller runs on a resource-constrained device with a real-time OS. In turn, a remote dashboard sends commands and receives updates to/from the controller. The goal of this assignment is to analyse the behaviour only of controller.

\begin{figure}[tb]
  \centering
  \begin{tikzpicture}
    \tikzstyle{nd}=[align=center,inner sep=7pt]
    \node[nd](db){
      \includegraphics[height=30pt]{img/dashboard.jpg}\\
      Dashboard};
    \node[nd,right=1.8 of db](cn){
      \includegraphics[height=30pt]{img/board.png}\\
      Controller};
    \node[nd,right=1.8 of cn](circ){
      \includegraphics[height=30pt]{img/motor.jpg}\\
      Circuit};
    \node[below] at (db.south) {};
    %
    % \draw[<->,black,very thick]
    %   (db)edge(cn) (pu)edge(circ);
    \draw[->,black,very thick,bend left=6,above,font=\sffamily]
      (db) edge node{commands} (cn)
      (cn) edge node{commands} (circ)
           edge node[below]{status} (db)
      (circ) edge node[below]{status} (cn);

    \pgfresetboundingbox
    \draw [draw=none] (db.north west) rectangle (circ.south east);
  \end{tikzpicture}
  \caption{Architecture of the motor controller system under verification}
  \label{fig:global-architecture}
\end{figure}

The overall behaviour of the controller is summarised in \cref{fig:motor-controllor-beh}.
Initially ...

\begin{figure}[tb]
  \centering
  \begin{tikzpicture}
    \tikzstyle{nd}=[circle,white,fill=green!70!black,
      align=center,very thick, minimum width=12mm,inner sep=1pt,
      font=\bfseries\sffamily]
    \node[nd](idle){Idle};
    \node[nd,right=1.5 of idle](ready){Ready};
    \coordinate[right=1.5 of ready](move);
    \node[nd,above=0.5 of move](left){Left};
    \node[nd,below=0.5 of move](right){Right};
    \node[nd,right=1.5 of move](error){Fall-\\[-2pt]back};
    %
    \coordinate[xshift=-10pt](start)at(idle.west);
    \coordinate[yshift=3pt](top)at(left.north);
    \coordinate[yshift=-3pt](bot)at(right.south);

    \tikzstyle{arr}=[->,thick,font=\scriptsize\sffamily,sloped,
                     inner sep=1pt,rounded corners=8pt]
    \tikzstyle{err}=[color=red!75!black]
    \draw[arr]
      (idle) edge[<-] (start)
      % (init)edge node[above]{all-ok}(idle)
      (idle)edge node[above,black](start){start}(ready)
      (ready) edge node[above]{left}(left)
              edge node[above]{right}(right)
              edge[err] node[above]{error}(error)
      (left)  edge[err] node[above]{error}(error)
      (right) edge[err] node[above]{error}(error);
    \draw[arr] (error) |- (bot) node[below]{reset} -| (idle);
    \draw[arr,err] (idle) |- (top) node[above]{error} -| (error);
    % \draw[arr] (idle) |- (top)                    -| (error);
    % \draw[arr,red] (st) -| (start) |- (top)      -| (error);
    \draw[arr] (left) -| node[above,pos=0.4]{stop} (ready) ;
    \draw[arr] (right) -| node[below,pos=0.4]{stop} (ready) ;
    % \draw[arr,bend right] (idle) edge (st)
                          % (st) edge (idle);
    % \draw[arr] (idle) edge[white,-,line width=4pt] (ready) edge (ready);

    \pgfresetboundingbox
    \draw [draw=none] (idle|-top) rectangle (error|-bot);
  \end{tikzpicture}
  \caption{General behaviour of the controller component}
  \label{fig:motor-controllor-beh}
\end{figure}




% \[\includegraphics[width=\textwidth]{RFID-conveyor.pdf}\]

\begin{myExercise} \label{ex:req}
  Rewrite the requirements below following the EARS approach [note to self: use colours to emphasise channels and components]
\end{myExercise}



\begin{myExercise} \label{ex:model}
  Model a concrete scenario of this controller as an UPPAAL model, i.e., with ... of your choice.
Use an automata for each ..., and
parameterise the model by (at least) the following constants:
\begin{itemize}
  \item Maximum attempts...;
  \item Minimum and maximum time at each phase;
\end{itemize}
% Note that there may be several approaches, some more detailed than others, to model this system. 
Justify clearly your decisions (assumptions and abstractions) made in this modelling exercise.
\end{myExercise}


\begin{myExercise} \label{ex:ctl}
  Use UPPAAL's CTL logic to express and verify properties.
  \subex{ \label{ex:ctl-req} Specify all properties from the requirements in \cref{ex:req}.}

  \subex{Express a property in UPPAAL's CTL logic for each item below. Fix a particular set of parameters (c.f. \cref{ex:model}), and say if each property holds or not (or if it takes too much time), and explain why.
  \begin{enumerate}
    \item ...
    \item ...
    \item ...
  \end{enumerate}
  }

  \subex{For each property from \cref{ex:ctl-req} find different values for the parameters, which satisfy and and reject them. If a given property is always true of false, justify informally why.}

\end{myExercise}




 
\end{document}