\documentclass[11pt]{article}


\input{macros}
\date{FVOCA -- 2021/2022}

\usepackage[latin1]{inputenc} % pt characters

% \usepackage{draftwatermark}
% \SetWatermarkText{Draft}
% \SetWatermarkScale{1.4}

\begin{document}
 
% --------------------------------------------------------------
%                         Start here
% --------------------------------------------------------------
 
\title{A0: Some assignment}

\author{David Pereira \& Jos\'{e} Proen\c{c}a \& Eduardo Tovar} 


\maketitle

% \vspace*{-2mm}
\descrbox{To do}{
  Produce a report as a PDF document including the answers to the exercises below.
}

\descrbox{What to submit}{
  The PDF report. 
}

\descrbox{How to submit using git}{
  \begin{enumerate}[itemsep=1pt,leftmargin=20pt]
    \item Create a private git repository in your favouring host (e.g., github or bitbucket).
    \item \textbf{Name it \bash{FVOCA-g<group number>}.}
    \item Add \bash{pro@isep.ipp.pt} as a member to the group (read-permissions are enough).
    \item Include all the files to be submitted in the repository.
  \end{enumerate}
  Note that \textbf{all students should push commits.}
}

\descrbox{Deadline}{
  dd mm yyyy @ 23:59 (Sunday)
}



\vspace{3mm}

\section*{First part}

Intro...

% \vspace*{-3mm}

\begin{myExercise} \label{ex:ex1}
Exercise one...

\subex{
First part
}

\subex{
Second part
}

\end{myExercise}


% \section*{Self-peer-evaluation}
% \begin{myExercise}
%   In a scale from 0-5, where 5 is better than 0, give a mark to you and each of your team groups for each of the following criteria:
%   \begin{itemize}
%     \item \textbf{Effort} (time spent)
%     \item \textbf{Quality} (of the work produced)
%     \item \textbf{Collaboration} (how easy it was to meet and interact)
%   \end{itemize}
%   \textbf{Send this information individually} by e-mail or via private message in Teams to David Pereira and Jos\'{e} Proen\c{c}a. No justification is needed -- e.g., \emph{``Group 3: Jo\~{a}o: Effort 5, Quality 4, Collaboration 5; Maria: Effort 4, Quality 4, Collaboration 4; ...''}.
% \end{myExercise}


\end{document}