%\documentclass[handout]{beamer}
\documentclass[aspectratio=169]{beamer}
\usepackage{etex} % fixes new-dimension error

\input{macros/beamerconf}
\input{macros/preamble}
\input{macros/macros}

%----------------------------------------------------------------------------
\usepackage{graphicx,amsmath}
\usepackage{stmaryrd} % cf. interleave
\usepackage{booktabs}
%\usepackage{./macros/myisolatin1}
\usepackage{amscd}

\usepackage{alltt}
%------ using xy ------------------------------------------------------------
\usepackage[all]{xy}
%\def\larrow#1#2#3{\xymatrix{ #3 & #1 \ar[l] _-{#2} }}
\def\larrow#1#2#3{\xymatrix{ #3 & #1 \ar[l] _--{#2} }}
\def\rarrow#1#2#3{\xymatrix{ #1 \ar[r]^-{#2} & #3 }}
\def\arLaw#1#2#3#4#5{
\xymatrix{
        #1      \ar@/^1pc/[rr]^-{#4} &
        #5 &
        #2      \ar@/^1pc/[ll]^-{#3}
}}
\def\arLeq#1#2#3#4{\arLaw{#1}{#2}{#3}{#4}\leq}
%------ using pstricks (rnode etc) ------------------------------------------
% \usepackage{pstricks,pst-node,pst-text,pst-3d}
%------ using color ---------------------------------------------------------

\definecolor{goldenrod}{rgb}{.80392 .60784 .11373}
\definecolor{darkgoldenrod}{rgb}{.5451 .39608 .03137}
\definecolor{brown}{rgb}{.15 .15 .15}
\definecolor{darkolivegreen}{rgb}{.33333 .41961 .18431}
%
%
\def\gold#1{{\goldenrod #1}}
% \def\alert#1{{\darkgoldenrod #1}}
% \def\alert#1{{\alert{#1}}}
%\def\brw#1{{\brown #1}}
% \def\structure#1{{\blue #1}}
% \def\tstructure#1{\textbf{\darkblue #1}}
%%\def\gre#1{{\green #1}}
\def\gre#1{{\darkolivegreen #1}}
\def\gry#1{{\textcolor{gray}{#1}}}
\def\rdb#1{{\red #1}}
\def\st{\mathbf{.}\,}
\def\laplace#1#2{*\txt{\mbox{ \fcolorbox{black}{myGray}{$\begin{array}{c}\mbox{#1}\\\\#2\\\\\end{array}$} }}}
%\newcommand{\galois}[2]{#1\; \dashv\; #2}

\def\eqm{\mathbin{\equiv}}                     
\def\noeqm{\mathbin{\not\!\equiv}}  
%\newcommand{\flam}[2]{\lambda_{#1}\; .\; #2}
\def\existential#1#2{\exists_{#1}\;.\; #2}
\def\existencial#1#2{\exists_{#1}\;.\; #2}

\def\pv#1#2{\langle #1 \rangle #2}
\def\nc#1#2{[#1]#2}
\def\pvo#1#2{\langle \! \! \! \langle #1 \rangle \! \! \! \rangle\, #2}
\def\nco#1#2{\llbracket #1 \rrbracket #2}
\def\cvg#1{\llbracket \downarrow \rrbracket #1}
\def\cvgr#1#2{\llbracket #1 \downarrow \rrbracket #2}
\def\cvgl#1#2{\llbracket \downarrow  #1 \rrbracket #2}
\def\cvglr#1#2{\llbracket \downarrow  #1 \downarrow \rrbracket #2}
\def\lfp#1#2{\mu {#1}\, .\, {#2}}
\def\lpf#1#2{\mu {#1}\, .\, {#2}}
\def\gfp#1#2{\nu {#1}\, .\, {#2}}
\def\gpf#1#2{\nu {#1}\, .\, {#2}}
\def\mset#1{\vvv #1 \vvv}
\def\vvv{\vert \! \vert}
\def\mnc#1{\vvv [#1] \vvv}
\def\mpv#1{\vvv \langle #1 \rangle \vvv}
\def\bcomp#1{#1^{\text{c}}}
\def\eqm{\mathbin{\simeq}}
\def\noeqm{\mathbin{\not\!\simeq}}
\def\universal#1#2{\forall_{#1}\;.\; #2}
\def\existential#1#2{\exists_{#1}\;.\; #2}
\def\oexistential#1#2{\exists^{1}_{#1}\;.\; #2}
\def\MM{\mathcal{M}}
\def\uppaal{\textsc{Uppaal}}
\def\cc#1{\mathcal{C}(#1)}
\def\R{\mathcal{R}}
\def\TL#1{\mathcal{T}(#1)}
\def\ET#1{\mathsf{ExecTime(#1)}}



\begin{document}

\setLectureBasic{Formal Verification of Critical Applications}
\frame[plain]{\titlepage}

% \title{
%   Time-critical reactive systems \\ (modelling )  }
% \frame[plain]{\titlepage}

%slides de motivacao a retirar se aparecerem primeiro os do mCRL2

% \section{Motivation}
% \begin{frame}[label=conclusion, standout]{Conclusion}
% Awesome slide
% \end{frame}

%----------------------------------------------------------------------------------
\begin{slide}{Last semester @ RAMDE}\label{xxx}
\small

  \frsplitt{
  \begin{itemize}
    \item High-level overview of requirements and associated processes
    \\[5mm]
    \item Mathematical Preliminaries
    \begin{itemize}
      \item Basic mathematical notations
      \item Set theory
      \item PropositionalLogic
      % \begin{itemize}
      %   \item Syntax, semantics, and reasoning
      % \end{itemize}
      \item \alert{First Order Logic}
      % \begin{itemize}
      %   \item Syntax, semantics, and reasoning
      % \end{itemize}
      % \item The Z3 automatic theorem prover
      % \begin{itemize}
      %   \item Rise4fun interface: get acquainted with the tool
      %   \item Python API: automating search for solutions
      % \end{itemize}
    \end{itemize}
  \end{itemize}
  }{
  \begin{itemize}
    \item Behavioural modelling with \alert{mCRL2}
    \begin{itemize}
      {
      \item \emph{Process algebra}
      % \item Single component
      % \begin{itemize}
      %   \item State diagrams and Flow charts
      %   \item Formal modelling: Automata, \alert{Process Algebra in mCRL2}
      % \end{itemize}
      % \item Many components
  %     \begin{itemize}
  %       \item Communication diagrams and Sequence
  % diagrams
  %       \item Formal modelling: Process algebra with interactions
  %     \end{itemize}
      }
      \item Equivalences
      % \begin{itemize}
      %   \item (Bi)similarity
      %   \item Realisability
      % \end{itemize}
      \item \alert{Verification}
      % \begin{itemize}
      %   \item Verification of requirements
      %   \item Formal modelling: modal logics
      %   \item Tools: model checking with mCRL2
      % \end{itemize}
    \end{itemize}

    \item Requirement analysis with \alert{EARS}
  \end{itemize}
  }
\end{slide}


%\end{document}
% fim slides motivacao


\begin{slide}{Course structure}
  \centering
  ~\\[0mm]

  \begin{enumerate}
    \item \textbf{Real-time models}
    \begin{itemize}
      \item Timed Automata and Hybrid Automata
      \item Temporal logic
      \item Static verification using \alert{UPPAAL}
    \end{itemize}

    \item \textbf{Program verification}
    \begin{itemize}
      \item First Order Logic revisited
      \item Abstract Program Semantics
      \item Design by Contract and Hoare Logic
      \item Verification of annotated programs
      % \item Runtime Verification
    \end{itemize}

    \item \textbf{Requirements}
    \begin{itemize}
      \item SAT and SMT solvers 
      \item \emph{Automatic} theorem proving using \alert{Z3}
      % \item Dependent types and certified programming
      \item Introduction to \emph{Interactive} theorem proving using \alert{Coq}
    \end{itemize}
  \end{enumerate}

\end{slide}



\begin{slide}{Pragmatics}
  \bigskip

  \myblock{\url{https://cister- labs.github.io/fvoca2122}}

  \begin{block}{Final mark = Project (60\%) + Exam (40\%)}
    \begin{itemize}
      \item Groups of 2 students
      \item Project in 2 parts
      \item Homework's evaluation included in the project
    \end{itemize}
  \end{block}

  \begin{block}{The Team}
    \frsplit{
    \begin{itemize}
    \item David Pereira (drp)
    \item José Proença (pro)
    \item Eduardo Tovar (emt)
    \end{itemize}
    }{
    \begin{itemize}
    \item Microsoft Teams
    \item Email (@isep.ipp.pt)
    \end{itemize}
    }
  \end{block}
\end{slide}



%%%%%%%%%%%%%%%%%%%%%%%%%%

\end{document}
